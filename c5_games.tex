\chapter{Games of the tournament}

\section{Game No.~1}
\begin{center}
Queen's Pawn Opening \\
\end{center} 
\begin{multicols}{2}
\noindent White \hfill Black \\
\noindent Dus--Chotimirski \hfill Mieses

\newgame

\noindent\mainline[style=styleC,level=1]{1. d4 Nf6 2. c4 d6 3. Nc3 Nbd7 4. e3}

%Best way I've found to print the chessboard in this style
%vspace is because the board adds an inappropriate amount of whitespace
\begin{center}
\vspace{-0.5cm}
\chessboard[smallboard,showmover=false]
\vspace{-0.1cm}
\end{center}

\noindent
After \variation[invar]{4. e4} the continuation might be: \variation{4... e5 5. Nf3 g6 6. Bg5 h6 7. Bh4 Bg7 8. Bg3 }

\noindent\mainline[outvar]{4... e5 5. Bd3 g6 6. f4 Qe7 7. Nge2 e4}

\noindent
A premature attempt at attack. \variation[invar]{7...Bg7} followed by \variation{8...O-O} and using the f8 rook on the e-file, was indicated. 

\mainline[outvar]{8. Bb1 c6 9. Qc2 Nb6 10. b3 Bf5 11. a4 }

\noindent
\variation[invar]{11. Ng3} would have been simply met by \variation{11... O-O-O }

\begin{center}
\vspace{-0.5cm}
\chessboard[smallboard,showmover=false]
\vspace{-0.1cm}
\end{center}

\mainline[outvar]{11... Rc8 12. a5 Na8 13. Ba3 Qe6 14. Qd2 d5 15. Bxf8 Kxf8 16. cxd5}

\noindent
This exchange was unnecessary. White ought to have continued at once with \variation[invar]{16. Na4} ; if then \variation{16... dxc4 17. Nc5} would follow with an excellent game. 

\begin{center}
\vspace{-0.5cm}
\chessboard[smallboard,showmover=false]
\vspace{-0.1cm}
\end{center}

\mainline[outvar]{16... cxd5 17. Na4 Kg7 18. O-O Nc7 19. Nc5 Qc6 20. Rc1 Qb5 21. Nc3 Qc6 22. Ne2}

\noindent
White might very well have continued \variation[invar]{22. b4} threatening to bring the light-squared bishop into action via c2 and a4; a plausible continuation would have been \variation{22... b6 23. axb6 axb6 24. N5a4 Nb5 25. Ne2 Qd6 26. h3} and white has a slight advantage.

\mainline[outvar]{22... Qb5 23. Nc3 Qc6 24. Ne2 Qb5 25. Nc3 Qc6 26. Na2 Qb5 27. Nc3 Qc6 28. Ne2 Qb5}

\begin{center}
Drawn\\
\end{center} 
\noindent 1h 15 \hfill 1h 15 \\
\begin{center}
\vspace{-.75cm}\noindent\rule{3cm}{0.4pt}
\end{center}
\end{multicols}


%%%%%%%%%%%%%%%%%%%%%%%%%%%%%%%%%%%%

%\vspace{0.5cm}
\section{Game No.~2}
\begin{center}
Vienna Game \\
\end{center} 
\begin{multicols}{2}
\noindent White \hfill Black \\
\noindent E. Cohn \hfill Burn

\newgame

\noindent\mainline[style=styleC,level=1]{1. e4 e5 2. Nc3 Bc5 3. g3 Nf6 4. Bg2 d6}

\begin{center}
\vspace*{-1cm}
\chessboard[smallboard,showmover=false]
\vspace{-0.1cm}
\end{center}

\noindent
\variation[invar]{4... Nc6} appears to be preferable, with a view to saving the important dark-squared bishop from being exchanged, by \variation{5...a6 }

\mainline[outvar]{5. Na4 Nc6 6. Ne2 Qe7 7. d3 Be6 8. O-O d5 9. Nxc5 Qxc5 10. Be3 Qd6 11. exd5 Bxd5 12. Nc3 Bxg2 13. Kxg2 Nd5 14. Qd2}

\noindent
\variation[invar]{14. Qf3} taking posession of the diagonal which the fianchettoed bishop commanded before, seems more natural.

\mainline[outvar]{14... O-O}

\noindent
Black ought to castle queenside, to attack on the kingside. 

\mainline{15. Ne4}

\begin{center}
\vspace{-0.5cm}
\chessboard[smallboard,showmover=false]
\vspace{-0.1cm}
\end{center}

\mainline{15... Nxe3+}

\noindent
\variation[invar]{15... Qg6 16. f4 f5 17. Nc3} (or \variation{17. Nc5 Nxe3+ 18. Qxe3 Nd4
17... Rad8}) would have created interesting complications, which would probably have turned out in Black's favour. 

\mainline[outvar]{16. Qxe3 Qd4 17. c3 Qxe3 18. fxe3}

\begin{center}
Drawn\\
\end{center} 
\noindent 1h 10 \hfill 0h 40 \\
\begin{center}
\vspace{-.75cm}\noindent\rule{3cm}{0.4pt}
\end{center}

\end{multicols}


%%%%%%%%%%%%%%%%%%%%%%%%%%%%%%%%%%%%

%\newpage
\section{Game No.~3}
\begin{center}
Queen's Gambit Declined \\
\end{center} 
\begin{multicols}{2}
\noindent White \hfill Black \\
\noindent Nenarokow \hfill Dr.Perlis

\newgame

\noindent\mainline[style=styleC,level=1]{1. d4 d5 2. c4 e6 3. Nc3 Nf6 4. Nf3 Be7 5. Bf4 O-O 6. e3 c5 7. Bd3 Nc6 8. cxd5 exd5 9. dxc5 Bxc5 10. O-O Be6 11. Rc1 Rc8}

\begin{center}
\vspace{-0.5cm}
\chessboard[smallboard,showmover=false]
\vspace{-0.1cm}
\end{center} 


\noindent 
Better \variation[invar]{11... a6 12. Bb1 d4 13. Na4 Ba7} ; the black dark-squared bishop should exert a pressure on d4. 

\mainline[outvar]{12. Bb1 Na5}

\noindent
There the knight is out of play. \variation[invar]{12... Qe7 13. Bg5 Rfd8 14. Qd3 h6} was a feasible line of play. The checks would have done Black no harm. 

\mainline[outvar]{13. Bg5 Be7 14. Nd4 g6}

\begin{center}
\vspace{-0.5cm}
\chessboard[smallboard,showmover=false]
\vspace{-0.1cm}
\end{center} 

\mainline{15. Qe2}

\noindent 
White might have played f4 followed by f5; e.g. \variation[invar]{15. f4 Bg4 16. Qe1 Nc4 17. f5 Nxb2 18. h3} and White would have an irresistible attack. 

\mainline[outvar]{15... Nc6 16. Nf3 Qb6 17. h3 Rfd8 18. Rfd1 Kg7 19. Nd4 Nxd4 20. exd4 Rc4 21. Be3 Rdc8 22. Bd3 Rb4 23. b3 Qd8 24. Na4 Rxc1 25. Rxc1 Bd7 26. Nc5 Rb6 27. Bf4 Bxc5 28. dxc5 Re6 29. Qb2 Qe7 30. Bd6 Qe8 31. Qd2 Bc6 32. Bf4 Ng8 33. Qc3+ f6 34. Kh2 Kf7 35. Qd2 a6 36. Bd6 Kg7 37. Bf4 Qe7}

\noindent
Adjourned. 

\begin{center}
\vspace{-0.5cm}
\chessboard[smallboard,showmover=false]
\vspace{-0.1cm}
\end{center}

\mainline{38. Bd6 Qe8 39. Bf4 Qe7 40. b4 Qe8 41. a3 Kf7 42. Rb1 f5 43. Rb2 Nf6 44. Bb1 Qe7 45. f3 Nh5 46. Bd6 Qh4 47. g3}

\noindent
Both parties have taken care not to alter the position to any considerable extent. Black here lays a trap. If \variation[invar]{47. Qh6} Black would have answered \variation{47... Rxd6}

\mainline[outvar]{47... Qd8 48. Ba2 Nf6 49. Kg2 Qe8 50. Kf2 Kg7 51. Bf4 Bb5}

\noindent
An altogether faulty manoeuvre; the attack thus imitated is easily parried, whilst the d-pawn is left without support. 

\mainline{52. Bh6+ Kh8 53. Qd1 Ng8 54. Qd4+ Nf6 55. h4}

\noindent
This was calculated to a nicety. 

\begin{center}
\vspace{-0.5cm}
\chessboard[smallboard,showmover=false]
\vspace{-0.1cm}
\end{center}

\mainline{55... Re2+ 56. Kg1 Re1+ 57. Kh2 Re2+ 58. Kh3 Qe6 59. Bg5 f4+ 60. g4 Re5}

\noindent
Black here lost the game by exceeding the time limit. The game might have gone on as follows: \variation[invar]{60... Re5 61. Qxf4 Bf1+ 62. Kh2 Nd7 63. Qd4} to White's advantage. 

\begin{center}
\vspace{-.75cm}\noindent\rule{3cm}{0.4pt}
\end{center}

\end{multicols}
