\section{Game No.~7}
\begin{center}
Queen's Gambit Declined \\
\end{center} 
\begin{multicols}{2}
\noindent White \hfill Black \\
\noindent Rubinstein \hfill Znosko--Borovsky

\newgame

\noindent\mainline[style=styleC,level=1]{1. d4 d5 2. c4 e6 3. Nc3 Nf6 4. Bg5 Be7 5. e3 Nbd7 6. Nf3 O-O 7. Qc2 b6 8. cxd5 exd5 9. Bd3 Bb7 10. O-O-O Ne4 11. h4 f5 12. Kb1 c5}

\begin{center}
\vspace{-0.5cm}
\chessboard[smallboard,showmover=false]
\vspace{-0.1cm}
\end{center} 

\noindent
\variation[invar]{12... Rc8} Should have been played instead. \variation{13. Qb3} would then be met simply by \variation{13... Nxc3+} and \variationNM{c5} 

\mainline[outvar]{13. dxc5 bxc5}

\noindent
After \variation[invar]{13... Ndxc5} White continues \variation{14. Nxd5 Bxd5 15. Bc4}. In this variation Black must not be able to take the bishop at d3 with a check, hence White's 12th move. 

\noindent
After \variation{13... Ndxc5 14. Nxd5 Bxg5} White would win by \variation{15. Bc4}

\begin{center}
\vspace{-0.5cm}
\chessboard[smallboard,showmover=false]
\vspace{-0.1cm}
\end{center}

\mainline[outvar]{14. Nxe4 fxe4 15. Bxe4 dxe4 16. Qb3+ Kh8 17. Qxb7 exf3 18. Rxd7 Qe8 19. Rxe7 Qg6+ 20. Ka1 Rab8 21. Qe4}

\noindent
White calculates every possibility with the utmost accuracy. 

\mainline{21... Qxe4 22. Rxe4 fxg2 23. Rg1 Rxf2 24. Rf4 Rc2}

\begin{center}
\vspace{-0.5cm}
\chessboard[smallboard,showmover=false]
\vspace{-0.1cm}
\end{center}

\noindent
If \variation[invar]{24... Rbxb2} White wins by \variation{25. Rf8+ }

\mainline[outvar]{25. b3 h6 26. Be7 Re8 27. Kb1 Re2 28. Bxc5 Rd8 29. Bd4 Rc8 30. Rg4}

\noindent
Black resigns.\\
%No game time info in the book
\begin{center}
\vspace{-.5cm}
\noindent 1h 47 \hspace{2cm} 2h 00 \\
\vspace{-.25cm}\noindent\rule{3cm}{0.4pt}
\end{center}

\vfill\null


\end{multicols}